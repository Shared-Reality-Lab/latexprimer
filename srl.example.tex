\documentclass{sigchi}

% Use this section to set the ACM copyright statement (e.g. for
% preprints).  Consult the conference website for the camera-ready
% copyright statement.

% Copyright
\CopyrightYear{2016}
%\setcopyright{acmcopyright}
\setcopyright{acmlicensed}
%\setcopyright{rightsretained}
%\setcopyright{usgov}
%\setcopyright{usgovmixed}
%\setcopyright{cagov}
%\setcopyright{cagovmixed}
% DOI
\doi{http://dx.doi.org/10.475/123_4}
% ISBN
\isbn{123-4567-24-567/08/06}
%Conference
\conferenceinfo{CHI'16,}{May 07--12, 2016, San Jose, CA, USA}
%Price
\acmPrice{\$15.00}

% Use this command to override the default ACM copyright statement
% (e.g. for preprints).  Consult the conference website for the
% camera-ready copyright statement.

%% HOW TO OVERRIDE THE DEFAULT COPYRIGHT STRIP --
%% Please note you need to make sure the copy for your specific
%% license is used here!
% \toappear{
% Permission to make digital or hard copies of all or part of this work
% for personal or classroom use is granted without fee provided that
% copies are not made or distributed for profit or commercial advantage
% and that copies bear this notice and the full citation on the first
% page. Copyrights for components of this work owned by others than ACM
% must be honored. Abstracting with credit is permitted. To copy
% otherwise, or republish, to post on servers or to redistribute to
% lists, requires prior specific permission and/or a fee. Request
% permissions from \href{mailto:Permissions@acm.org}{Permissions@acm.org}. \\
% \emph{CHI '16},  May 07--12, 2016, San Jose, CA, USA \\
% ACM xxx-x-xxxx-xxxx-x/xx/xx\ldots \$15.00 \\
% DOI: \url{http://dx.doi.org/xx.xxxx/xxxxxxx.xxxxxxx}
% }

% Arabic page numbers for submission.  Remove this line to eliminate
% page numbers for the camera ready copy
% \pagenumbering{arabic}

% Load basic packages
\usepackage{balance}       % to better equalize the last page
\usepackage{graphics}      % for EPS, load graphicx instead 
\usepackage[T1]{fontenc}   % for umlauts and other diaeresis
\usepackage{txfonts}
\usepackage{mathptmx}
\usepackage[pdflang={en-US},pdftex]{hyperref}
\usepackage{color}
\usepackage{booktabs}
\usepackage{textcomp}

% Some optional stuff you might like/need.
\usepackage{microtype}        % Improved Tracking and Kerning
% \usepackage[all]{hypcap}    % Fixes bug in hyperref caption linking
\usepackage{ccicons}          % Cite your images correctly!
% \usepackage[utf8]{inputenc} % for a UTF8 editor only

% If you want to use todo notes, marginpars etc. during creation of
% your draft document, you have to enable the "chi_draft" option for
% the document class. To do this, change the very first line to:
% "\documentclass[chi_draft]{sigchi}". You can then place todo notes
% by using the "\todo{...}"  command. Make sure to disable the draft
% option again before submitting your final document.
\usepackage{todonotes}

% Paper metadata (use plain text, for PDF inclusion and later
% re-using, if desired).  Use \emtpyauthor when submitting for review
% so you remain anonymous.
\def\plaintitle{Using LaTeX at SRL}
\def\plainauthor{Jeff Blum, Second Author, Third Author,
  Fourth Author, Fifth Author, Sixth Author}
\def\emptyauthor{}
\def\plainkeywords{Authors' choice; of terms; separated; by
  semicolons; include commas, within terms only; required.}
\def\plaingeneralterms{Documentation, Standardization}

% llt: Define a global style for URLs, rather that the default one
\makeatletter
\def\url@leostyle{%
  \@ifundefined{selectfont}{
    \def\UrlFont{\sf}
  }{
    \def\UrlFont{\small\bf\ttfamily}
  }}
\makeatother
\urlstyle{leo}

% To make various LaTeX processors do the right thing with page size.
\def\pprw{8.5in}
\def\pprh{11in}
\special{papersize=\pprw,\pprh}
\setlength{\paperwidth}{\pprw}
\setlength{\paperheight}{\pprh}
\setlength{\pdfpagewidth}{\pprw}
\setlength{\pdfpageheight}{\pprh}

% Make sure hyperref comes last of your loaded packages, to give it a
% fighting chance of not being over-written, since its job is to
% redefine many LaTeX commands.
\definecolor{linkColor}{RGB}{6,125,233}
\hypersetup{%
  pdftitle={\plaintitle},
% Use \plainauthor for final version.
%  pdfauthor={\plainauthor},
  pdfauthor={\emptyauthor},
  pdfkeywords={\plainkeywords},
  pdfdisplaydoctitle=true, % For Accessibility
  bookmarksnumbered,
  pdfstartview={FitH},
  colorlinks,
  citecolor=black,
  filecolor=black,
  linkcolor=black,
  urlcolor=linkColor,
  breaklinks=true,
  hypertexnames=false
}

% create a shortcut to typeset table headings
% \newcommand\tabhead[1]{\small\textbf{#1}}

% SRL commands to control commenting/blinding/etc
%\newcommand*{\srlhidecomments}{}%   %put this line (uncommented) in your document to hide comments
%\newcommand*{\srlblind{}%   %put this line (uncommented)in your document to turn blinding on
%%% SRL packages and commands %%%
%20170215	jeffbl	initial version

% copy srl.tex into your own project, then copy/paste the below lines into your document just before \begin{document}
% BEGIN COPY PASTE SECTION
% SRL commands to control commenting/blinding/etc
%\newcommand*{\srlhidecomments}{}%   %put this line (uncommented) in your document to hide comments
%\newcommand*{\srlblind{}%   %put this line (uncommented)in your document to turn blinding on
%%%% SRL packages and commands %%%
%20170215	jeffbl	initial version

% copy srl.tex into your own project, then copy/paste the below lines into your document just before \begin{document}
% BEGIN COPY PASTE SECTION
% SRL commands to control commenting/blinding/etc
%\newcommand*{\srlhidecomments}{}%   %put this line (uncommented) in your document to hide comments
%\newcommand*{\srlblind{}%   %put this line (uncommented)in your document to turn blinding on
%%%% SRL packages and commands %%%
%20170215	jeffbl	initial version

% copy srl.tex into your own project, then copy/paste the below lines into your document just before \begin{document}
% BEGIN COPY PASTE SECTION
% SRL commands to control commenting/blinding/etc
%\newcommand*{\srlhidecomments}{}%   %put this line (uncommented) in your document to hide comments
%\newcommand*{\srlblind{}%   %put this line (uncommented)in your document to turn blinding on
%\input{srl.tex} %uncomment this to include the file so you get all the nice commands...
% END COPY PASTE SECTION


\usepackage{color,soul}
\usepackage{booktabs} %much nicer tables
\usepackage{subfig} %for putting multiple subimages in one figure, with individual captions

% Put things in \hide{this text is hidden!} to make sure they don't appear
\newcommand{\hide}[1]{}

%Commenting commands
\sethlcolor{yellow}
\ifdefined\srlhidecomments
	\newcommand{\jer}[1] {}
	\newcommand{\jef}[1] {}
	\newcommand{\jan}[1] {}
    \newcommand{\pef}[1] {}
    %add your own name here...
\else
	\definecolor{burntorange}{rgb}{0.8, 0.33, 0.0}
    %and your own color here if you want a custom one...something not too light...
    % custom color definitions at http://latexcolor.com/

	\newcommand{\jer}[1] { \textcolor{red}{[{Jer:} {#1}}]}
	\newcommand{\jef}[1] { \textcolor{burntorange}{[{\hl{Jef:}} {#1}}]}
	\newcommand{\jan}[1] { \textcolor{green}{[{Jan:} {#1}}]}
    \newcommand{\pef}[1] { \textcolor{magenta}{[{Pef:} {#1}}]}
    %and don't forget to add your name here!
\fi

%blinding for review
\ifdefined\srlblind
	\newcommand{\blind}[1]{[omitted for blind review]}
\else
	\newcommand{\blind}[1]{#1} %for camera ready (not blinded)
\fi

%%% end SRL packages and commands %%% %uncomment this to include the file so you get all the nice commands...
% END COPY PASTE SECTION


\usepackage{color,soul}
\usepackage{booktabs} %much nicer tables
\usepackage{subfig} %for putting multiple subimages in one figure, with individual captions

% Put things in \hide{this text is hidden!} to make sure they don't appear
\newcommand{\hide}[1]{}

%Commenting commands
\sethlcolor{yellow}
\ifdefined\srlhidecomments
	\newcommand{\jer}[1] {}
	\newcommand{\jef}[1] {}
	\newcommand{\jan}[1] {}
    \newcommand{\pef}[1] {}
    %add your own name here...
\else
	\definecolor{burntorange}{rgb}{0.8, 0.33, 0.0}
    %and your own color here if you want a custom one...something not too light...
    % custom color definitions at http://latexcolor.com/

	\newcommand{\jer}[1] { \textcolor{red}{[{Jer:} {#1}}]}
	\newcommand{\jef}[1] { \textcolor{burntorange}{[{\hl{Jef:}} {#1}}]}
	\newcommand{\jan}[1] { \textcolor{green}{[{Jan:} {#1}}]}
    \newcommand{\pef}[1] { \textcolor{magenta}{[{Pef:} {#1}}]}
    %and don't forget to add your name here!
\fi

%blinding for review
\ifdefined\srlblind
	\newcommand{\blind}[1]{[omitted for blind review]}
\else
	\newcommand{\blind}[1]{#1} %for camera ready (not blinded)
\fi

%%% end SRL packages and commands %%% %uncomment this to include the file so you get all the nice commands...
% END COPY PASTE SECTION


\usepackage{color,soul}
\usepackage{booktabs} %much nicer tables
\usepackage{subfig} %for putting multiple subimages in one figure, with individual captions

% Put things in \hide{this text is hidden!} to make sure they don't appear
\newcommand{\hide}[1]{}

%Commenting commands
\sethlcolor{yellow}
\ifdefined\srlhidecomments
	\newcommand{\jer}[1] {}
	\newcommand{\jef}[1] {}
	\newcommand{\jan}[1] {}
    \newcommand{\pef}[1] {}
    %add your own name here...
\else
	\definecolor{burntorange}{rgb}{0.8, 0.33, 0.0}
    %and your own color here if you want a custom one...something not too light...
    % custom color definitions at http://latexcolor.com/

	\newcommand{\jer}[1] { \textcolor{red}{[{Jer:} {#1}}]}
	\newcommand{\jef}[1] { \textcolor{burntorange}{[{\hl{Jef:}} {#1}}]}
	\newcommand{\jan}[1] { \textcolor{green}{[{Jan:} {#1}}]}
    \newcommand{\pef}[1] { \textcolor{magenta}{[{Pef:} {#1}}]}
    %and don't forget to add your name here!
\fi

%blinding for review
\ifdefined\srlblind
	\newcommand{\blind}[1]{[omitted for blind review]}
\else
	\newcommand{\blind}[1]{#1} %for camera ready (not blinded)
\fi

%%% end SRL packages and commands %%% %uncomment this to include this file so you get all the nice commands...

% End of preamble. Here it comes the document.
\begin{document}

\title{\plaintitle}

\numberofauthors{3}
\author{%
  \alignauthor{Jeff Blum\\
    \affaddr{for Submission}\\
    \affaddr{City, Country}\\
    \email{e-mail address}}\\
  \alignauthor{Leave Authors Anonymous\\
    \affaddr{for Submission}\\
    \affaddr{City, Country}\\
    \email{e-mail address}}\\
  \alignauthor{Leave Authors Anonymous\\
    \affaddr{for Submission}\\
    \affaddr{City, Country}\\
    \email{e-mail address}}\\
}

\maketitle

\begin{abstract}
  A sample document that demonstrates srl.tex and gives specific advice for how we do papers, plus general tips and tricks...
\end{abstract}

\category{H.5.m.}{Information Interfaces and Presentation
  (e.g. HCI)}{Miscellaneous} \category{See
  \url{http://acm.org/about/class/1998/} for the full list of ACM
  classifiers. This section is required.}{}{}

\keywords{\plainkeywords}

\section{Introduction}

Welcome to SRL.
We use LaTeX.
Please don't use Word!

\section{LaTeX tools}
It looks like we're standardizing on using Overleaf\footnote{\url{http://ovewrleaf.com}} for collaborative online editing, but this doesn't work when you're offline, so you'll likely want to set up a real repository via git, and a LaTeX IDE.
Jeff likes texmaker\footnote{\url{http://www.xm1math.net/texmaker/}} on Linux, but there are others.
If you do use it on (for example) Ubuntu, the following packages should get pretty much everything you need:

\begin{verbatim}
sudo apt-get install \
texmaker \
texlive-fonts-recommended \
texlive-fonts-extra \
texlive-pictures \
texlive-latex-extra \
texlive-bibtex-extra \
imagemagick
\end{verbatim}

\section{Getting Started}
Don't make your own document from scratch.
Go get the latest SIGCHI (Jeff's preference) or IEEE template, chop out (after reading it!) the demonstration text in the example file, and start adding your own content there.
Even if you don't know where you'll eventually submit, it is better to start with a real template, and change it later.

\section{SRL commands in srl.tex}

\subsection{commenting}
For comments, you use a commenting command based on your name, which you have to set up in srl.tex.
Put it in the master version so that everyone gets it for free when they update their srl.tex file.
Note that before the command there is no space, since otherwise when you shut comments off to send camera ready copy, things will format funny\jef{here is an example comment, note no space before the command in the LaTeX code!}.
Uncomment the srlhidecomments command at the top of your document to hide all the comments, e.g., before you submit the PDF!
Blinding the authors can be more tricky, and may need to be done by hand.
\jef{Anyone know of a good way to do this?}

\subsection{hide}
If there are things you want to effectively comment out, use the hide command.
This should be the last sentence you see in this section in the PDF, but there is actually a hide line right after this.
\hide{This does not appear in the compiled PDF!}

\subsection{blinding}
Most conferences want you to anonymize your publication for review, so that nobody knows who wrote it.
If you use the blind command, your secret text will be \blind{normal like this}, but when you uncomment the srlblind command at the top of the document, they will magically become [omitted for blind review].
Thus, you can toggle it on and off for the entire document to get your pagination and such correct.

\section{Top 10 SRL tips and tricks}

\begin{enumerate}

\item
Compile your document often.
NEVER check in a new version without first compiling it.
If you don't check, you'll forget a brace somewhere, and everyone else working on the document will hate you.

\item
Don't ever use the built in latex tables.
They are super ugly.
Use booktabs instead.\footnote{\url{http://cs.brown.edu/about/system/managed/latex/doc/booktabs.pdf}}
Note that the package is already included in srl.tex.

\item
Put each sentence on its own line.
If you don't, you'll be very sad when trying to do merges via source control (svn/git).
This won't bother you until everyone is editing at once, which usually happens right before deadline, and then you will curse yourself for not listening.

\item
Before you use the ref command to do a citation, use a tilde to make a non-breaking space, so that your citation index does not bump to the next line.
\jef{Needs example.}

\end{enumerate}

\section{Overleaf and git}

TBD

\section{Tutorial notes and stuff to add later}

Before we do a live tutorial, please use either overleaf or a local install (e.g., texmaker) to create a simple LaTeX document and compile it to PDF.
There are many online tutorials for this (suggestions?).

Overleaf/git integration - main is our git. (confidentiality?) put overleaf link in the document as a comment?

tables \& booktabs

figures/images : placement on page, column widths, subfig, full width, labeling and referring

tweaking space (vspace)

math/equations

common special characters (e.g.,  degree symbol)

final bib tweaks - e.g., condensing conference/journal names

cleaning up for submitting LaTeX source

noindent

bibtex/mendeley

what to do if can't find command? google it and install the package. on ubuntu at least, do *not* try and manually install individual packages except as a last resort - figure out which texlive package it is in, and just install the whole thing

Advanced things not covered : draft mode, 

% Balancing columns in a ref list is a bit of a pain because you
% either use a hack like flushend or balance, or manually insert
% a column break.  http://www.tex.ac.uk/cgi-bin/texfaq2html?label=balance
% multicols doesn't work because we're already in two-column mode,
% and flushend isn't awesome, so I choose balance.  See this
% for more info: http://cs.brown.edu/system/software/latex/doc/balance.pdf
%
% Note that in a perfect world balance wants to be in the first
% column of the last page.
%
% If balance doesn't work for you, you can remove that and
% hard-code a column break into the bbl file right before you
% submit:
%
% http://stackoverflow.com/questions/2149854/how-to-manually-equalize-columns-
% in-an-ieee-paper-if-using-bibtex
%
% Or, just remove \balance and give up on balancing the last page.
%
\balance{}

% BALANCE COLUMNS
\balance{}

% REFERENCES FORMAT
% References must be the same font size as other body text.
\bibliographystyle{SIGCHI-Reference-Format}
\bibliography{sample}

\end{document}

%%% Local Variables:
%%% mode: latex
%%% TeX-master: t
%%% End:
