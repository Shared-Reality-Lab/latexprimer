\documentclass{sigchi}

% Use this section to set the ACM copyright statement (e.g. for
% preprints).  Consult the conference website for the camera-ready
% copyright statement.

% Copyright
\CopyrightYear{2016}
%\setcopyright{acmcopyright}
\setcopyright{acmlicensed}
%\setcopyright{rightsretained}
%\setcopyright{usgov}
%\setcopyright{usgovmixed}
%\setcopyright{cagov}
%\setcopyright{cagovmixed}
% DOI
\doi{http://dx.doi.org/10.475/123_4}
% ISBN
\isbn{123-4567-24-567/08/06}
%Conference
\conferenceinfo{CHI'16,}{May 07--12, 2016, San Jose, CA, USA}
%Price
\acmPrice{\$15.00}

% Use this command to override the default ACM copyright statement
% (e.g. for preprints).  Consult the conference website for the
% camera-ready copyright statement.

%% HOW TO OVERRIDE THE DEFAULT COPYRIGHT STRIP --
%% Please note you need to make sure the copy for your specific
%% license is used here!
% \toappear{
% Permission to make digital or hard copies of all or part of this work
% for personal or classroom use is granted without fee provided that
% copies are not made or distributed for profit or commercial advantage
% and that copies bear this notice and the full citation on the first
% page. Copyrights for components of this work owned by others than ACM
% must be honored. Abstracting with credit is permitted. To copy
% otherwise, or republish, to post on servers or to redistribute to
% lists, requires prior specific permission and/or a fee. Request
% permissions from \href{mailto:Permissions@acm.org}{Permissions@acm.org}. \\
% \emph{CHI '16},  May 07--12, 2016, San Jose, CA, USA \\
% ACM xxx-x-xxxx-xxxx-x/xx/xx\ldots \$15.00 \\
% DOI: \url{http://dx.doi.org/xx.xxxx/xxxxxxx.xxxxxxx}
% }

% Arabic page numbers for submission.  Remove this line to eliminate
% page numbers for the camera ready copy
% \pagenumbering{arabic}

% Load basic packages
\usepackage{balance}       % to better equalize the last page
\usepackage{graphics}      % for EPS, load graphicx instead 
\usepackage[T1]{fontenc}   % for umlauts and other diaeresis
\usepackage{txfonts}
\usepackage{mathptmx}
\usepackage[pdflang={en-US},pdftex]{hyperref}
\usepackage{color}
\usepackage{booktabs}
\usepackage{textcomp}

% Some optional stuff you might like/need.
\usepackage{microtype}        % Improved Tracking and Kerning
% \usepackage[all]{hypcap}    % Fixes bug in hyperref caption linking
\usepackage{ccicons}          % Cite your images correctly!
% \usepackage[utf8]{inputenc} % for a UTF8 editor only

% If you want to use todo notes, marginpars etc. during creation of
% your draft document, you have to enable the "chi_draft" option for
% the document class. To do this, change the very first line to:
% "\documentclass[chi_draft]{sigchi}". You can then place todo notes
% by using the "\todo{...}"  command. Make sure to disable the draft
% option again before submitting your final document.
\usepackage{todonotes}

% Paper metadata (use plain text, for PDF inclusion and later
% re-using, if desired).  Use \emtpyauthor when submitting for review
% so you remain anonymous.
\def\plaintitle{Using LaTeX at SRL}
\def\plainauthor{Jeff Blum, Second Author, Third Author,
  Fourth Author, Fifth Author, Sixth Author}
\def\emptyauthor{}
\def\plainkeywords{Authors' choice; of terms; separated; by
  semicolons; include commas, within terms only; required.}
\def\plaingeneralterms{Documentation, Standardization}

% llt: Define a global style for URLs, rather that the default one
\makeatletter
\def\url@leostyle{%
  \@ifundefined{selectfont}{
    \def\UrlFont{\sf}
  }{
    \def\UrlFont{\small\bf\ttfamily}
  }}
\makeatother
\urlstyle{leo}

% To make various LaTeX processors do the right thing with page size.
\def\pprw{8.5in}
\def\pprh{11in}
\special{papersize=\pprw,\pprh}
\setlength{\paperwidth}{\pprw}
\setlength{\paperheight}{\pprh}
\setlength{\pdfpagewidth}{\pprw}
\setlength{\pdfpageheight}{\pprh}

% Make sure hyperref comes last of your loaded packages, to give it a
% fighting chance of not being over-written, since its job is to
% redefine many LaTeX commands.
\definecolor{linkColor}{RGB}{6,125,233}
\hypersetup{%
  pdftitle={\plaintitle},
% Use \plainauthor for final version.
%  pdfauthor={\plainauthor},
  pdfauthor={\emptyauthor},
  pdfkeywords={\plainkeywords},
  pdfdisplaydoctitle=true, % For Accessibility
  bookmarksnumbered,
  pdfstartview={FitH},
  colorlinks,
  citecolor=black,
  filecolor=black,
  linkcolor=black,
  urlcolor=linkColor,
  breaklinks=true,
  hypertexnames=false
}

% create a shortcut to typeset table headings
% \newcommand\tabhead[1]{\small\textbf{#1}}

% SRL commands to control commenting/blinding/etc
%\newcommand*{\srlhidecomments}{}%   %put this line (uncommented) in your document to hide comments
%\newcommand*{\srlblind}{}%   %put this line (uncommented)in your document to turn blinding on
%%% SRL packages and commands %%%
%20170215   jeffbl  initial version

% copy srl.tex into your own project, then copy/paste the below lines into your document just before \begin{document}
% BEGIN COPY PASTE SECTION
% SRL commands to control commenting/blinding/etc
%\newcommand*{\srlhidecomments}{}%   %put this line (uncommented) in your document to hide comments
%\newcommand*{\srlblind}{}%   %put this line (uncommented)in your document to turn blinding on
%%%% SRL packages and commands %%%
%20170215   jeffbl  initial version

% copy srl.tex into your own project, then copy/paste the below lines into your document just before \begin{document}
% BEGIN COPY PASTE SECTION
% SRL commands to control commenting/blinding/etc
%\newcommand*{\srlhidecomments}{}%   %put this line (uncommented) in your document to hide comments
%\newcommand*{\srlblind}{}%   %put this line (uncommented)in your document to turn blinding on
%%%% SRL packages and commands %%%
%20170215   jeffbl  initial version

% copy srl.tex into your own project, then copy/paste the below lines into your document just before \begin{document}
% BEGIN COPY PASTE SECTION
% SRL commands to control commenting/blinding/etc
%\newcommand*{\srlhidecomments}{}%   %put this line (uncommented) in your document to hide comments
%\newcommand*{\srlblind}{}%   %put this line (uncommented)in your document to turn blinding on
%\input{srl.tex} %uncomment this to include the file so you get all the nice commands...
% END COPY PASTE SECTION


\usepackage{color,soul}
\usepackage{booktabs} %much nicer tables
\usepackage{subfig} %for putting multiple subimages in one figure, with individual captions
\usepackage{graphicx}
\usepackage{hyperref}
\hypersetup{
    colorlinks   = true, %Colours links instead of ugly boxes
    urlcolor     = blue, %Colour for external hyperlinks
    linkcolor    = blue, %Colour of internal links
    citecolor   = red %Colour of citations
}
%\usepackage{siunitx}
\usepackage[inter-unit-product =\cdot]{siunitx}
\newcommand{\mcgill}{Department of Electrical and Computer Engineering, McGill University, Montr\'eal, Canada }
\newcommand\foreign[1]{\emph{#1}}
% Put things in \hide{this text is hidden!} to make sure they don't appear
\newcommand{\hide}[1]{}

%Commenting commands
\sethlcolor{yellow}
\ifdefined\srlhidecomments
    \newcommand{\jer}[1] {}
    \newcommand{\jef}[1] {}
    \newcommand{\jan}[1] {}
    \newcommand{\pef}[1] {}
    \newcommand{\dan}[1] {}
    \newcommand{\guo}[1] {}
    \newcommand{\mot}[1] {}
    \newcommand{\taeyong}[1]{}
    \newcommand{\hao}[1] {}
    \newcommand{\paris}[1] {}
    \newcommand{\rog}[1] {}
    \newcommand{\joe}[1] {}
    \newcommand{\tord}[1] {}
    \newcommand{\gan}[1] {}
    \newcommand{\lis}[1] {}
    \newcommand{\zored}[1] {}
    \newcommand{\Catherine}[1] {}
    \newcommand{\reb}[1] {}
    \newcommand{\bij}[1] {}
    \newcommand{\fer}[1] {}
    \newcommand{\ani}[1] {}
    \newcommand{\pre}[1] {}
    \newcommand{\oli}[1] {}
    \newcommand{\andre}[1] {}
    \newcommand{\rev}[1] {}
    \newcommand{\jam}[1] {}
    \newcommand{\ant}[1] {}
    \newcommand{\marc}[1] {}

    %add your own name here...
\else
    \definecolor{burntorange}{rgb}{0.8, 0.33, 0.0}
    \definecolor{cadmiumgreen}{rgb}{0.0, 0.42, 0.24}
    \definecolor{cobalt}{rgb}{0.0, 0.28, 0.67}
    \definecolor{amber}{rgb}{1.0, 0.75, 0.0}
    \definecolor{fashionfuchsia}{rgb}{0.96, 0.0, 0.63}
    \definecolor{brightcerulean}{rgb}{0.11, 0.67, 0.84}
    \definecolor{frenchblue}{rgb}{0.0, 0.45, 0.73}
    %and your own color here if you want a custom one...something not too light...
    % custom color definitions at http://latexcolor.com/

    \newcommand{\jer}[1] { \textcolor{red}{[\hl{Jer:} {#1}}]}
    \newcommand{\jef}[1] { \textcolor{burntorange}{[{\hl{Jef:}} {#1}}]}
    \newcommand{\pef}[1] { \textcolor{magenta}{[{Pef:} {#1}}]}
    \newcommand{\dan}[1] { \textcolor{green}{[{Dan:} {#1}}]}
    \newcommand{\guo}[1] { \textcolor{blue}{[{Guo:} {#1}}]} 
    \newcommand{\taeyong}[1]{\textcolor{blue}{[{Taeyong:}{#1}}]}
    \newcommand{\hao}[1]{\textcolor{red!55!blue}{[{Hao:}{#1}}]}
    \newcommand{\gan}[1]{\textcolor{burntorange}{[{Gan:}{#1}}]}
    \newcommand{\paris}[1] { \textcolor{cadmiumgreen}{[{Par:} {#1}}]}
    \newcommand{\rog}[1] { \textcolor{red!55!yellow}{[{Rog:} {#1}}]}
    \newcommand{\joe}[1] { \textcolor{blue}{[{Joe:} {#1}}]}
    \newcommand{\tord}[1] { \textcolor{cyan}{[{tord:} {#1}}]}
    \newcommand{\mot}[1] { \textcolor{cyan}{[{Mot:} {#1}}]}
    \newcommand{\jan}[1] { \textcolor[rgb]{0.2,0.3,0.8}{[{Jan:} {#1}}]}
    \newcommand{\zored}[1] { \textcolor{teal}{[{Zor:} {#1}}]}
    \newcommand{\lis}[1] { \textcolor{magenta}{[{Lis:} {#1}}]}
    \newcommand{\Catherine}[1] { \textcolor{purple}{[{Catherine:} {#1}}]}
    \newcommand{\pxe}[1] { \textcolor{cobalt}{[{Pre:} {#1}}]}
    \newcommand{\reb}[1] { \textcolor{green}{[{Reb:} {#1}}]}
    \newcommand{\bij}[1] { \textcolor{pink}{[{Bij:} {#1}}]}
    \newcommand{\ani}[1] { \textcolor{blue}{[{Ani:} {#1}}]}
    \newcommand{\fer}[1] { \textcolor{amber}{[{Fer:} {#1}}]}
    \newcommand{\pre}[1] { \textcolor{green}{[{Pre:} {#1}}]}
    \newcommand{\oli}[1] { \textcolor{amber}{[{Oliver:} {#1}}]}
    \newcommand{\andre}[1] { \textcolor{fashionfuchsia}{[{Andre:} {#1}}]}
    \newcommand{\rev}[1] { \textcolor{brightcerulean}{[{Reviewer Feedback:} {#1}}]}
    \newcommand{\jam}[1] { \textcolor{cobalt}{[{jam:} {#1}}]}
    \newcommand{\ant}[1] { \textcolor{frenchblue}{[{\hl{Antoine:}} {#1}]}}
    \newcommand{\marc}[1] { \textcolor{cyan}{[{\hl{Marc:}} {#1}]}}
    %and don't forget to add your name here!
% Balancing columns
    
\fi

%blinding for review
\ifdefined\srlblind
    \newcommand{\blind}[1]{[omitted for blind review]}
\else
    \newcommand{\blind}[1]{#1} %for camera ready (not blinded)
\fi

%%% end SRL packages and commands %%% %uncomment this to include the file so you get all the nice commands...
% END COPY PASTE SECTION


\usepackage{color,soul}
\usepackage{booktabs} %much nicer tables
\usepackage{subfig} %for putting multiple subimages in one figure, with individual captions
\usepackage{graphicx}
\usepackage{hyperref}
\hypersetup{
    colorlinks   = true, %Colours links instead of ugly boxes
    urlcolor     = blue, %Colour for external hyperlinks
    linkcolor    = blue, %Colour of internal links
    citecolor   = red %Colour of citations
}
%\usepackage{siunitx}
\usepackage[inter-unit-product =\cdot]{siunitx}
\newcommand{\mcgill}{Department of Electrical and Computer Engineering, McGill University, Montr\'eal, Canada }
\newcommand\foreign[1]{\emph{#1}}
% Put things in \hide{this text is hidden!} to make sure they don't appear
\newcommand{\hide}[1]{}

%Commenting commands
\sethlcolor{yellow}
\ifdefined\srlhidecomments
    \newcommand{\jer}[1] {}
    \newcommand{\jef}[1] {}
    \newcommand{\jan}[1] {}
    \newcommand{\pef}[1] {}
    \newcommand{\dan}[1] {}
    \newcommand{\guo}[1] {}
    \newcommand{\mot}[1] {}
    \newcommand{\taeyong}[1]{}
    \newcommand{\hao}[1] {}
    \newcommand{\paris}[1] {}
    \newcommand{\rog}[1] {}
    \newcommand{\joe}[1] {}
    \newcommand{\tord}[1] {}
    \newcommand{\gan}[1] {}
    \newcommand{\lis}[1] {}
    \newcommand{\zored}[1] {}
    \newcommand{\Catherine}[1] {}
    \newcommand{\reb}[1] {}
    \newcommand{\bij}[1] {}
    \newcommand{\fer}[1] {}
    \newcommand{\ani}[1] {}
    \newcommand{\pre}[1] {}
    \newcommand{\oli}[1] {}
    \newcommand{\andre}[1] {}
    \newcommand{\rev}[1] {}
    \newcommand{\jam}[1] {}
    \newcommand{\ant}[1] {}
    \newcommand{\marc}[1] {}

    %add your own name here...
\else
    \definecolor{burntorange}{rgb}{0.8, 0.33, 0.0}
    \definecolor{cadmiumgreen}{rgb}{0.0, 0.42, 0.24}
    \definecolor{cobalt}{rgb}{0.0, 0.28, 0.67}
    \definecolor{amber}{rgb}{1.0, 0.75, 0.0}
    \definecolor{fashionfuchsia}{rgb}{0.96, 0.0, 0.63}
    \definecolor{brightcerulean}{rgb}{0.11, 0.67, 0.84}
    \definecolor{frenchblue}{rgb}{0.0, 0.45, 0.73}
    %and your own color here if you want a custom one...something not too light...
    % custom color definitions at http://latexcolor.com/

    \newcommand{\jer}[1] { \textcolor{red}{[\hl{Jer:} {#1}}]}
    \newcommand{\jef}[1] { \textcolor{burntorange}{[{\hl{Jef:}} {#1}}]}
    \newcommand{\pef}[1] { \textcolor{magenta}{[{Pef:} {#1}}]}
    \newcommand{\dan}[1] { \textcolor{green}{[{Dan:} {#1}}]}
    \newcommand{\guo}[1] { \textcolor{blue}{[{Guo:} {#1}}]} 
    \newcommand{\taeyong}[1]{\textcolor{blue}{[{Taeyong:}{#1}}]}
    \newcommand{\hao}[1]{\textcolor{red!55!blue}{[{Hao:}{#1}}]}
    \newcommand{\gan}[1]{\textcolor{burntorange}{[{Gan:}{#1}}]}
    \newcommand{\paris}[1] { \textcolor{cadmiumgreen}{[{Par:} {#1}}]}
    \newcommand{\rog}[1] { \textcolor{red!55!yellow}{[{Rog:} {#1}}]}
    \newcommand{\joe}[1] { \textcolor{blue}{[{Joe:} {#1}}]}
    \newcommand{\tord}[1] { \textcolor{cyan}{[{tord:} {#1}}]}
    \newcommand{\mot}[1] { \textcolor{cyan}{[{Mot:} {#1}}]}
    \newcommand{\jan}[1] { \textcolor[rgb]{0.2,0.3,0.8}{[{Jan:} {#1}}]}
    \newcommand{\zored}[1] { \textcolor{teal}{[{Zor:} {#1}}]}
    \newcommand{\lis}[1] { \textcolor{magenta}{[{Lis:} {#1}}]}
    \newcommand{\Catherine}[1] { \textcolor{purple}{[{Catherine:} {#1}}]}
    \newcommand{\pxe}[1] { \textcolor{cobalt}{[{Pre:} {#1}}]}
    \newcommand{\reb}[1] { \textcolor{green}{[{Reb:} {#1}}]}
    \newcommand{\bij}[1] { \textcolor{pink}{[{Bij:} {#1}}]}
    \newcommand{\ani}[1] { \textcolor{blue}{[{Ani:} {#1}}]}
    \newcommand{\fer}[1] { \textcolor{amber}{[{Fer:} {#1}}]}
    \newcommand{\pre}[1] { \textcolor{green}{[{Pre:} {#1}}]}
    \newcommand{\oli}[1] { \textcolor{amber}{[{Oliver:} {#1}}]}
    \newcommand{\andre}[1] { \textcolor{fashionfuchsia}{[{Andre:} {#1}}]}
    \newcommand{\rev}[1] { \textcolor{brightcerulean}{[{Reviewer Feedback:} {#1}}]}
    \newcommand{\jam}[1] { \textcolor{cobalt}{[{jam:} {#1}}]}
    \newcommand{\ant}[1] { \textcolor{frenchblue}{[{\hl{Antoine:}} {#1}]}}
    \newcommand{\marc}[1] { \textcolor{cyan}{[{\hl{Marc:}} {#1}]}}
    %and don't forget to add your name here!
% Balancing columns
    
\fi

%blinding for review
\ifdefined\srlblind
    \newcommand{\blind}[1]{[omitted for blind review]}
\else
    \newcommand{\blind}[1]{#1} %for camera ready (not blinded)
\fi

%%% end SRL packages and commands %%% %uncomment this to include the file so you get all the nice commands...
% END COPY PASTE SECTION


\usepackage{color,soul}
\usepackage{booktabs} %much nicer tables
\usepackage{subfig} %for putting multiple subimages in one figure, with individual captions
\usepackage{graphicx}
\usepackage{hyperref}
\hypersetup{
    colorlinks   = true, %Colours links instead of ugly boxes
    urlcolor     = blue, %Colour for external hyperlinks
    linkcolor    = blue, %Colour of internal links
    citecolor   = red %Colour of citations
}
%\usepackage{siunitx}
\usepackage[inter-unit-product =\cdot]{siunitx}
\newcommand{\mcgill}{Department of Electrical and Computer Engineering, McGill University, Montr\'eal, Canada }
\newcommand\foreign[1]{\emph{#1}}
% Put things in \hide{this text is hidden!} to make sure they don't appear
\newcommand{\hide}[1]{}

%Commenting commands
\sethlcolor{yellow}
\ifdefined\srlhidecomments
    \newcommand{\jer}[1] {}
    \newcommand{\jef}[1] {}
    \newcommand{\jan}[1] {}
    \newcommand{\pef}[1] {}
    \newcommand{\dan}[1] {}
    \newcommand{\guo}[1] {}
    \newcommand{\mot}[1] {}
    \newcommand{\taeyong}[1]{}
    \newcommand{\hao}[1] {}
    \newcommand{\paris}[1] {}
    \newcommand{\rog}[1] {}
    \newcommand{\joe}[1] {}
    \newcommand{\tord}[1] {}
    \newcommand{\gan}[1] {}
    \newcommand{\lis}[1] {}
    \newcommand{\zored}[1] {}
    \newcommand{\Catherine}[1] {}
    \newcommand{\reb}[1] {}
    \newcommand{\bij}[1] {}
    \newcommand{\fer}[1] {}
    \newcommand{\ani}[1] {}
    \newcommand{\pre}[1] {}
    \newcommand{\oli}[1] {}
    \newcommand{\andre}[1] {}
    \newcommand{\rev}[1] {}
    \newcommand{\jam}[1] {}
    \newcommand{\ant}[1] {}
    \newcommand{\marc}[1] {}

    %add your own name here...
\else
    \definecolor{burntorange}{rgb}{0.8, 0.33, 0.0}
    \definecolor{cadmiumgreen}{rgb}{0.0, 0.42, 0.24}
    \definecolor{cobalt}{rgb}{0.0, 0.28, 0.67}
    \definecolor{amber}{rgb}{1.0, 0.75, 0.0}
    \definecolor{fashionfuchsia}{rgb}{0.96, 0.0, 0.63}
    \definecolor{brightcerulean}{rgb}{0.11, 0.67, 0.84}
    \definecolor{frenchblue}{rgb}{0.0, 0.45, 0.73}
    %and your own color here if you want a custom one...something not too light...
    % custom color definitions at http://latexcolor.com/

    \newcommand{\jer}[1] { \textcolor{red}{[\hl{Jer:} {#1}}]}
    \newcommand{\jef}[1] { \textcolor{burntorange}{[{\hl{Jef:}} {#1}}]}
    \newcommand{\pef}[1] { \textcolor{magenta}{[{Pef:} {#1}}]}
    \newcommand{\dan}[1] { \textcolor{green}{[{Dan:} {#1}}]}
    \newcommand{\guo}[1] { \textcolor{blue}{[{Guo:} {#1}}]} 
    \newcommand{\taeyong}[1]{\textcolor{blue}{[{Taeyong:}{#1}}]}
    \newcommand{\hao}[1]{\textcolor{red!55!blue}{[{Hao:}{#1}}]}
    \newcommand{\gan}[1]{\textcolor{burntorange}{[{Gan:}{#1}}]}
    \newcommand{\paris}[1] { \textcolor{cadmiumgreen}{[{Par:} {#1}}]}
    \newcommand{\rog}[1] { \textcolor{red!55!yellow}{[{Rog:} {#1}}]}
    \newcommand{\joe}[1] { \textcolor{blue}{[{Joe:} {#1}}]}
    \newcommand{\tord}[1] { \textcolor{cyan}{[{tord:} {#1}}]}
    \newcommand{\mot}[1] { \textcolor{cyan}{[{Mot:} {#1}}]}
    \newcommand{\jan}[1] { \textcolor[rgb]{0.2,0.3,0.8}{[{Jan:} {#1}}]}
    \newcommand{\zored}[1] { \textcolor{teal}{[{Zor:} {#1}}]}
    \newcommand{\lis}[1] { \textcolor{magenta}{[{Lis:} {#1}}]}
    \newcommand{\Catherine}[1] { \textcolor{purple}{[{Catherine:} {#1}}]}
    \newcommand{\pxe}[1] { \textcolor{cobalt}{[{Pre:} {#1}}]}
    \newcommand{\reb}[1] { \textcolor{green}{[{Reb:} {#1}}]}
    \newcommand{\bij}[1] { \textcolor{pink}{[{Bij:} {#1}}]}
    \newcommand{\ani}[1] { \textcolor{blue}{[{Ani:} {#1}}]}
    \newcommand{\fer}[1] { \textcolor{amber}{[{Fer:} {#1}}]}
    \newcommand{\pre}[1] { \textcolor{green}{[{Pre:} {#1}}]}
    \newcommand{\oli}[1] { \textcolor{amber}{[{Oliver:} {#1}}]}
    \newcommand{\andre}[1] { \textcolor{fashionfuchsia}{[{Andre:} {#1}}]}
    \newcommand{\rev}[1] { \textcolor{brightcerulean}{[{Reviewer Feedback:} {#1}}]}
    \newcommand{\jam}[1] { \textcolor{cobalt}{[{jam:} {#1}}]}
    \newcommand{\ant}[1] { \textcolor{frenchblue}{[{\hl{Antoine:}} {#1}]}}
    \newcommand{\marc}[1] { \textcolor{cyan}{[{\hl{Marc:}} {#1}]}}
    %and don't forget to add your name here!
% Balancing columns
    
\fi

%blinding for review
\ifdefined\srlblind
    \newcommand{\blind}[1]{[omitted for blind review]}
\else
    \newcommand{\blind}[1]{#1} %for camera ready (not blinded)
\fi

%%% end SRL packages and commands %%% %uncomment this to include this file so you get all the nice commands...

% End of preamble. Here it comes the document.
\begin{document}

\title{\plaintitle}

\numberofauthors{3}
\author{%
  \alignauthor{Jeff Blum\\
    \affaddr{for Submission}\\
    \affaddr{City, Country}\\
    \email{e-mail address}}\\
  \alignauthor{Leave Authors Anonymous\\
    \affaddr{for Submission}\\
    \affaddr{City, Country}\\
    \email{e-mail address}}\\
  \alignauthor{Leave Authors Anonymous\\
    \affaddr{for Submission}\\
    \affaddr{City, Country}\\
    \email{e-mail address}}\\
}

\maketitle
\begin{abstract}
  A sample document that demonstrates srl.tex and gives specific advice for how we do papers, plus general tips and tricks...
\end{abstract}

\category{H.5.m.}{Information Interfaces and Presentation
  (e.g. HCI)}{Miscellaneous} \category{See
  \url{http://acm.org/about/class/1998/} for the full list of ACM
  classifiers. This section is required.}{}{}

\keywords{\plainkeywords}

\section{Introduction}

Welcome to SRL.
We use LaTeX.
Please don't use Word!
And lord have mercy on your soul if you happen to be asked to submit to conferences that only accept Word documents (some even using IEEE format).

\section{LaTeX tools}
It looks like we're standardizing on using Overleaf\footnote{\url{http://ovewrleaf.com}} for collaborative online editing, but this doesn't work when you're offline, so you'll likely want to set up a real repository via git, and a LaTeX IDE.
Jeff likes texmaker\footnote{\url{http://www.xm1math.net/texmaker/}} on Linux, but there are others.
If you do use it on (for example) Ubuntu, the following packages should get pretty much everything you need:

\begin{verbatim}
sudo apt-get install \
texmaker \
texlive-fonts-recommended \
texlive-fonts-extra \
texlive-pictures \
texlive-latex-extra \
texlive-bibtex-extra \
imagemagick
\end{verbatim}

\section{Getting Started}
Don't make your own document from scratch.
Go get the latest SIGCHI (Jeff's preference) or IEEE template, chop out (after reading it!) the demonstration text in the example file, and start adding your own content there.
Even if you don't know where you'll eventually submit, it is better to start with a real template, and change it later.

If you are using Overleaf, you can create a link to the srl.tex file, so that you should automatically get changes as it is updated, without having to copy/paste it.
Note that you do, however, have to manually refresh it from within your project when you want updates.
To do this, go to Files... -> "File from other Project" in your project in Overleaf, and choose the srl.tex file from this project.
\footnote{\url{https://www.overleaf.com/help/226-can-i-share-files-e-dot-g-bib-and-some-graphics-across-my-projects}}

\section{SRL commands in srl.tex}



\subsection{commenting}
For comments, you use a commenting command based on your name, which you have to set up in srl.tex.
Put it in the master version so that everyone gets it for free when they update their srl.tex file.
Note that before the command there is no space, since otherwise when you shut comments off to send camera ready copy, things will format funny \jef{here is an example comment, note no space before the command in the LaTeX code!}.
Uncomment the srlhidecomments command at the top of your document to hide all the comments, e.g., before you submit the PDF!
Blinding the authors can be more tricky, and may need to be done by hand.
\jef{Anyone know of a good way to do this?}
\jer{It seems that you've already addressed this!}

\subsection{hide}
If there are things you want to effectively comment out, use the hide command.
This should be the last sentence you see in this section in the PDF, but there is actually a hide line right after this.
\hide{This does not appear in the compiled PDF!}

\subsection{blinding}
Most conferences want you to anonymize your publication for review, so that nobody knows who wrote it.
If you use the blind command, your secret text will be \blind{normal like this}, but when you uncomment the srlblind command at the top of the document, they will magically become [omitted for blind review].
Thus, you can toggle it on and off for the entire document to get your pagination and such correct.

\section{Top 10 SRL tips and tricks}

\begin{enumerate}

\item
Compile your document often.
NEVER check in a new version without first compiling it.
If you don't check, you'll forget a brace somewhere, and everyone else working on the document will hate you.
Same goes for missing included files, such as figures, that you did not include in your commit.

\item
Don't ever use the built in latex tables.
They are super ugly.
Use booktabs instead.\footnote{\url{http://cs.brown.edu/about/system/managed/latex/doc/booktabs.pdf}}
Note that the package is already included in srl.tex.
Wikibooks has a short tutorial that explains them fairly well.
\jer{an example would be nice; I've actually found the \LaTeX table and longtable environments perfectly adequate, so am curious to see the difference}\jef{There are examples in the footnoted URL, but you can also use booktabs with longtable...booktabs makes the tables publication-quality, and longtable makes splits across pages better. The footnoted link provides more information about using them together, and this link covers many tabular-related latex packages\footnote{\url{http://tex.stackexchange.com/questions/12672/which-tabular-packages-do-which-tasks-and-which-packages-conflict}}.}
\dan{There is also this handy website\footnote{\url{http://www.tablesgenerator.com/}} that facilitates table making. Don't forget to change the table style! Look for the drop-down menu at the far right of the bold/italic toolbar.}
\item
Put each sentence on its own line.
If you don't, you'll be very sad when trying to do merges via source control (svn/git).
This won't bother you until everyone is editing at once, which usually happens right before deadline, and then you will curse yourself for not listening.

\item
Before you use the ref command to do a citation, use a tilde to make a non-breaking space, so that your citation index does not bump to the next line.
\jef{Needs example.}

\end{enumerate}

\section{Overleaf and git}

As a general guideline, never commit generated files, such as PDFs or .aux (etc.) files - there is a .gitignore file at /home/srecollab/git/gitignoreLatex.

If you want to use both Overleaf and a local git, that is possible, as is putting everything into the lab git repository on barn.
Basic steps :

\begin{enumerate}
\item Git clone the repository from barn
\item Set a second remote to overleaf
\item git push to one or both remotes as you make changes
\end{enumerate}

I suggest that we put the overleaf URL in the document as a comment so we know where to find it.

Also, note that Overleaf probably is not a good place to put confidential (e.g., participant) data, so be careful.

\section{Citations}

This is a must to cover, at least briefly, since it's often done wrong.
Make sure to add your citations in an external .bib file such as the sample.bib here.
It doesn't matter whether you use a bib generator such as Mendeley or "do it yourself". 
Here's a citation to the Sensorama, an important piece of prior literature of relevance to Shared Reality~\cite{heilig:sensorama}.

\section{Tutorial notes and stuff to add later}

Before we do a live tutorial, please use either overleaf or a local install (e.g., texmaker) to create a simple LaTeX document and compile it to PDF.
There are many online tutorials for this (suggestions?).

For the tutorial, will use a lab UIST paper as an example:
\begin{verbatim}
git clone git+ssh://$USER@barn.cim.mcgill.ca/home/srecollab/git/publications/2015-CHI-jeffbl-HapticPerception.git
\end{verbatim}

figures/images : placement on page, column widths, subfig, full width, labeling and referring

tweaking space (vspace)

math/equations $x= \sum_a^b y^3$ Note that you can build them online graphically, e.g. \url{http://www.sciweavers.org/free-online-latex-equation-editor}

common special characters (e.g.,  degree symbol)

final bib tweaks - e.g., condensing conference/journal names

cleaning up for submitting LaTeX source

noindent

bibtex/mendeley

what to do if can't find command? google it and install the package. on ubuntu at least, do *not* try and manually install individual packages except as a last resort - figure out which texlive package it is in, and just install the whole thing (use \url{http://packages.ubuntu.com} for example and search for the package containing the filename).

Advanced things not covered : draft mode, 

% Balancing columns in a ref list is a bit of a pain because you
% either use a hack like flushend or balance, or manually insert
% a column break.  http://www.tex.ac.uk/cgi-bin/texfaq2html?label=balance
% multicols doesn't work because we're already in two-column mode,
% and flushend isn't awesome, so I choose balance.  See this
% for more info: http://cs.brown.edu/system/software/latex/doc/balance.pdf
%
% Note that in a perfect world balance wants to be in the first
% column of the last page.
%
% If balance doesn't work for you, you can remove that and
% hard-code a column break into the bbl file right before you
% submit:
%
% http://stackoverflow.com/questions/2149854/how-to-manually-equalize-columns-
% in-an-ieee-paper-if-using-bibtex
%
% Or, just remove \balance and give up on balancing the last page.
%
% BALANCE COLUMNS
\balance{}

% REFERENCES FORMAT
% References must be the same font size as other body text.
\bibliographystyle{SIGCHI-Reference-Format}
\bibliography{sample}

\end{document}

%%% Local Variables:
%%% mode: latex
%%% TeX-master: t
%%% End:
